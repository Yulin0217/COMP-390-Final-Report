\documentclass[12pt]{article}
\usepackage{geometry}
\geometry{a4paper}
\usepackage{graphicx} % Needed for including images
\usepackage{mdframed} % For creating framed boxes
\usepackage[backend=biber, style=ieee]{biblatex} % Adding biblatex with IEEE style
\addbibresource{Reference.bib} % Specify the bibliography file, here 'references.bib'


\begin{document}

\begin{titlepage}
    \centering


    \includegraphics[width=0.6\textwidth]{Liverpool.jpg} % Increased width
    \vspace*{1cm}

    \Large
    COMP390

    \large
    2023/24

    \vspace{0.5cm}
    \Huge
    \textbf{Computer Vision and AR for Endovascular Intervention}

    \vspace{1.5cm}


    % Framed box for student information
    \begin{mdframed}
        \normalsize % Smaller text size within the box
        \textbf{Student Name:} [Yulin Huang]\\[20pt] % Name on the same line, add vertical space
        \textbf{Student ID:} [201676465]\\[20pt] % ID on the same line, add vertical space
        \textbf{Supervisor Name:} [Anh Nguyen] % Supervisor on the same line
    \end{mdframed}

    \vspace{2cm} % Adjust space as necessary
    \Large
    \textbf{DEPARTMENT OF}\\
    \vspace{0.1cm} % Adjust line spacing
    \textbf{COMPUTER SCIENCE}

    \vspace{4cm} % Large space as required
    \large
    University of Liverpool\\
    Liverpool L69 3BX


\end{titlepage}
% Acknowledgements page
\newpage
\thispagestyle{empty} % Remove page number from the acknowledgements page
\begin{center}
    \Large \textbf{Acknowledgements}
\end{center}
\vspace{1cm}
\normalsize
%Here for the Acknowledgements

\newpage
\begin{titlepage}
    \centering


    \includegraphics[width=0.6\textwidth]{Liverpool.jpg} % Increased width
    \vspace*{1cm}

    \Large
    COMP390

    \large
    2023/24

    \vspace{4.5cm}
    \Huge
    \textbf{Computer Vision and AR for Endovascular Intervention}

    \vspace{1.5cm}



    \vspace{4cm} % Adjust space as necessary
    \Large
    \textbf{DEPARTMENT OF}\\
    \vspace{0.1cm} % Adjust line spacing
    \textbf{COMPUTER SCIENCE}

    \vspace{1cm} % Large space as required
    \large
    University of Liverpool\\
    Liverpool L69 3BX


\end{titlepage}
\tableofcontents
\newpage

% Abstract
\section*{Abstract}
This section should contain a concise summary of the document content.

% Statement of Ethical Compliance
\section*{Statement of Ethical Compliance}
\begin{mdframed}
    \Large % Larger font size for the first two lines
Data Category: A \\
Participant Category: 0 \\
    \normalsize % Normal font size for the rest of the text
    I confirm that I have read the ethical guidelines and will follow them during this project. Further details can be found in the relevant sections of this proposal.
\end{mdframed}
\newpage

% Introduction & Background
\section{Introduction \& Background}



\section{Design \& Implementation}

\subsection{Part1: Real-world Model Interaction and Tracking}
% 这一部分集中讨论与现实世界模型的交互与跟踪的设计和实现。

In this part, I used OpenCV\cite{opencv_library} and SciKit-Surgery Augmented Reality\cite{Thompson_SciKit-Surgery_Compact_Libraries_2020} libraries for image 
processing and  model tracking, while QT\cite{QtWebsite} is used to design the graphical interface. OpenCV and SciKit-Surgery help in processing images 
and tracking ArUco Markers\cite{1467495} within these images, essential for enhancing interaction between virtual and real-world elements. 
QT allows me to create a user-friendly interface, ensuring that the system is accessible and manageable for users with different levels of technical expertise. 
This section will detail the system's architecture and functionality, focusing on System Components and Organization, Data Structures and Algorithms, 
User Interface Design, and Design Notation and Diagrams.


\subsubsection{Design}
\begin{enumerate}
    \item \textbf{System Components and Organization}
          % 描述系统中各组件的组织和协作方式。

    \item \textbf{Data Structures and Algorithms}
          % 详细说明系统所使用的数据结构和算法。
          \begin{enumerate}
              \item \textbf{Frame Storage}
                    % 描述如何存储和处理连续的图像帧,例如使用循环缓冲区或队列来管理实时视频帧。

              \item \textbf{Feature Data Structures}
                    % 详细说明用于存储从视频中提取的特征(如ArUco标记的位置、角点检测等)的数据结构,可能涉及到空间哈希表或树结构(如KD树、四叉树)。

              \item \textbf{Image Processing Algorithms}
                    % 讨论用于视频帧处理的基本算法,如灰度转换、滤波、边缘检测等。

              \item \textbf{Marker Detection and Tracking}
                    % 详述用于检测和追踪ArUco标记的算法,包括图像分割、模式识别和机器学习技术。

              \item \textbf{Model Positioning and Rendering}
                    % 介绍如何将三维模型精确地放置在虚拟环境中,可能涉及计算几何和物体识别算法。

              \item \textbf{Optimization Techniques}
                    % 如果涉及到性能瓶颈,可以介绍用于优化上述算法的技术,如多线程处理、GPU加速或使用更高效的算法(如快速傅里叶变换代替传统卷积)。

              \item \textbf{Algorithm Efficiency and Complexity}
                    % 分析关键算法的时间和空间复杂度,讨论在不同操作环境(如不同硬件或软件平台)下的性能表现。
          \end{enumerate}


    \item \textbf{User Interface Design}
          \begin{enumerate}
              \item \textbf{Screen Mockups, Sketches, and Screenshots}
                    % 包括界面设计的屏幕模拟图、草图和截图。

              \item \textbf{OpenCV-Based Real-world Interaction Component}
                    % 描述该组件如何将模拟整合到AR/VR平台,以及实时视频捕捉与覆盖的技术细节。

              \item \textbf{ArUco Marker Detection and Model Positioning}
                    % 讨论如何在视频帧中自动检测ArUco标记,并实现模型在虚拟环境中的精确定位。
          \end{enumerate}


    \item \textbf{Design Notation and Diagrams}
          \begin{enumerate}
              \item \textbf{Use Case Diagrams}
                    % 描述用例图及其在设计中的应用。

              \item \textbf{Interaction Diagrams and Class Diagrams}
                    % 介绍交互图和类图的设计和功能。

              \item \textbf{Pseudocode and Data Flow Diagrams}
                    % 提供系统关键部分的伪代码及数据流图。
          \end{enumerate}
\end{enumerate}

\subsubsection{Implementation}
% 在这里描述 Part1: Real-world Model Interaction and Tracking 部分的具体实现细节。

\subsection{Part2: Endovascular Intervention Simulation}
% 在这里详细介绍关于“Endovascular Intervention Simulation”的设计和实现。

\subsubsection{Design}
% 描述“Endovascular Intervention Simulation”部分的设计细节。

\subsubsection{Implementation}
% 描述“Endovascular Intervention Simulation”部分的具体实现细节。


% Testing & Evaluation
\section{Testing \& Evaluation}

% Project Ethics
\section{Project Ethics}

% Conclusion & Future Work
\section{Conclusion \& Future Work}
\subsection{Conclusion}

\subsection{Future Work}

\section{BCS Criteria \& Self-Reflection}

\subsection{An Ability to Apply Practical and Analytical Skills}
% Describe how you applied practical and analytical skills gained during your degree programme to your project.

\subsection{Innovation and/or Creativity}
% Discuss any innovative or creative approaches employed in your project.

\subsection{Synthesis of Information, Ideas, and Practices}
% Explain how you synthesized information, ideas, and practices to provide a quality solution. Include an evaluation of that solution.

\subsection{Meeting a Real Need in a Wider Context}
% Reflect on how your project meets a real need in a wider context.

\subsection{An Ability to Self-Manage a Significant Piece of Work}
% Reflect on your ability to self-manage the significant piece of work that is your project.

\subsection{Critical Self-Evaluation of the Process}
% Provide a critical self-evaluation of the process involved in your project.


% References (The bibliography will be printed here)
\printbibliography
\end{document}
