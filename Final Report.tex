\documentclass[12pt]{article}
\usepackage{geometry}
\geometry{a4paper}
\usepackage[colorlinks, linkcolor=blue, citecolor=blue, urlcolor=blue]{hyperref}
\usepackage[automake]{glossaries-extra}
\usepackage{appendix}
\usepackage{graphicx} % Needed for including images
\usepackage{mdframed} % For creating framed boxes
\usepackage[backend=biber, style=ieee]{biblatex} % Adding biblatex with IEEE style
\usepackage{minted}

\addbibresource{Reference.bib} % Specify the bibliography file, here 'references.bib'


\makeglossaries % 初始化术语表系统

% 定义一些术语
 
\newglossaryentry{opencv}{
    name=OpenCV,
    description={OpenCV (Open Source Computer Vision Library) is a library of programming functions mainly for real-time computer vision. Originally developed by Intel, it was later supported by Willow Garage, then Itseez (which was later acquired by Intel). The library is cross-platform and licensed as free and open-source software under Apache License 2. Starting in 2011, OpenCV features GPU acceleration for real-time operations.\cite{qt}}
}

\newglossaryentry{QT}{
    name=Qt,
    description={Qt (pronounced "cute" or as an initialism) is cross-platform application development framework for creating graphical user interfaces as well as cross-platform applications that run on various software and hardware platforms such as Linux, Windows, macOS, Android or embedded systems with little or no change in the underlying codebase while still being a native application with native capabilities and speed.\cite{cv}}
}

\newglossaryentry{aruco}{
    name=Aruco Marker,
    description={ArUco markers are 2D binary-encoded fiducial patterns designed to be quickly located by computer vision systems. ArUco marker patterns are defined by a binary dictionary in OpenCV, and the various library functions return pattern IDs and pose information from scanned images.\cite{arucomarkers}}
}

\newglossaryentry{vtk}{
      name=VTK,
      description={The Visualization Toolkit (VTK) is an open-source, freely available software system for 3D computer graphics, modeling, image processing, volume rendering, scientific visualization, and 2D plotting. It supports a wide variety of visualization algorithms and advanced modeling techniques, and it takes advantage of both threaded and distributed memory parallel processing for speed and scalability, respectively.\cite{vtkBook}}
}

\newglossaryentry{alpha}{
      name=Alpha Blending,
      description={In computer graphics, alpha compositing or alpha blending is the process of combining one image with a background to create the appearance of partial or full transparency. It is often useful to render picture elements (pixels) in separate passes or layers and then combine the resulting 2D images into a single, final image called the composite. Compositing is used extensively in film when combining computer-rendered image elements with live footage. Alpha blending is also used in 2D computer graphics to put rasterized foreground elements over a background.\cite{Alpha}}
}

\newglossaryentry{aesl}{
      name=Adaptive Exponential Smoothing,
      description={Adaptive exponential smoothing models are designed to improve performance by letting the smoothing parameter vary according to the most recent forecasting accuracy. \cite{aes,aes2}}
}

\newglossaryentry{emal}{
      name=Exponential Moving Average,
      description={An exponential moving average (EMA) is a type of moving average (MA) that places a greater weight and significance on the most recent data points. The exponential moving average is also referred to as the exponentially weighted moving average. An exponentially weighted moving average reacts more significantly to recent price changes than a simple moving average simple moving average (SMA), which applies an equal weight to all observations in the period.\cite{InvestopediaEMA}}
}

\newglossaryentry{cesl}{
      name=Complex Exponential Smoothing,
      description={Complex exponential smoothing is a time series forecasting method that combines exponential smoothing with trend and seasonality. It is a variant of the standard exponential smoothing method, which is a simple technique for smoothing out data by using a weighted average of past observations.\cite{ComplexES2018}}
}


\newglossaryentry{mrtkl}{
      name=Microsoft Mixed Reality Toolkit-Unityde,
      description={MRTK-Unity is a Microsoft-driven project that provides a set of components and features, used to accelerate cross-platform MR app development in Unity. Here are some of its functions: 1.Provides the cross-platform input system and building blocks for spatial interactions and UI. 2.Enables rapid prototyping via in-editor simulation that allows you to see changes immediately. 3.Operates as an extensible framework that provides developers the ability to swap out core components.\cite{MRTKUnity2024}}
}

\newglossaryentry{blender}{
      name=Blender,
      description={Blender is a free and open-source 3D computer graphics software tool set used for creating animated films, visual effects, art, 3D-printed models, motion graphics, interactive 3D applications, virtual reality, and, formerly, video games. Blender's features include 3D modelling, UV mapping, texturing, digital drawing, raster graphics editing, rigging and skinning, fluid and smoke simulation, particle simulation, soft body simulation, sculpting, animation, match moving, rendering, motion graphics, video editing, and compositing\cite{Blender2024,blender}.}
}

\newglossaryentry{holo}{
      name=Microsoft Hololens,
      description={Microsoft HoloLens is an augmented reality (AR)/mixed reality (MR) headset developed and manufactured by Microsoft. HoloLens runs the Windows Mixed Reality platform under the Windows 10 operating system. Some of the positional tracking technology used in HoloLens can trace its lineage to the Microsoft Kinect, an accessory for Microsoft's Xbox 360 and Xbox One game consoles that was introduced in 2010\cite{holo1,holo2}}
}

\newglossaryentry{git}{
      name=GitHub,
      description={GitHub is a developer platform that allows developers to create, store, manage and share their code. It uses Git software, providing the distributed version control of Git plus access control, bug tracking, software feature requests, task management, continuous integration, and wikis for every project. Headquartered in California, it has been a subsidiary of Microsoft since 2018. It is commonly used to host open source software development projects. As of January 2023, GitHub reported having over 100 million developers and more than 420 million repositories, including at least 28 million public repositories. It is the world's largest source code host as of June 2023.\cite{git}}
}

% 定义一个缩写词
\newabbreviation{aes}{AES}{
      Adaptive Exponential Smoothing}

\newabbreviation{ema}{EMA}{
      Exponential Moving Average}

\newabbreviation{ces}{CES}{
      Complex Exponential Smoothing}    

\newabbreviation{mrtk}{MRTK}{
      Microsoft Mixed Reality Toolkit-Unity}    
      

\begin{document}

\begin{titlepage}
      \centering

      \includegraphics[width=0.6\textwidth]{Liverpool.jpg} % Increased width
      \vspace*{1cm}

      \Large
      COMP390

      \large
      2023/24

      \vspace{0.5cm}
      \Huge
      \textbf{Computer Vision and AR for Endovascular Intervention}

      \vspace{1.5cm}


      % Framed box for student information
      \begin{mdframed}
            \normalsize % Smaller text size within the box
            \textbf{Student Name:} [Yulin Huang]\\[20pt] % Name on the same line, add vertical space
            \textbf{Student ID:} [201676465]\\[20pt] % ID on the same line, add vertical space
            \textbf{Supervisor Name:} [Anh Nguyen] % Supervisor on the same line
      \end{mdframed}

      \vspace{2cm} % Adjust space as necessary
      \Large
      \textbf{DEPARTMENT OF}\\
      \vspace{0.1cm} % Adjust line spacing
      \textbf{COMPUTER SCIENCE}

      \vspace{4cm} % Large space as required
      \large
      University of Liverpool\\
      Liverpool L69 3BX


\end{titlepage}
% Acknowledgements page
\newpage
\thispagestyle{empty} % Remove page number from the acknowledgements page
\begin{center}
      \Large \textbf{Acknowledgements}
\end{center}
\vspace{1cm}
\normalsize
%Here for the Acknowledgements

\newpage
\begin{titlepage}
      \centering


      \includegraphics[width=0.6\textwidth]{Liverpool.jpg} % Increased width
      \vspace*{1cm}

      \Large
      COMP390
      \large
      2023/24

      \vspace{4.5cm}
      \Huge
      \textbf{Computer Vision and AR for Endovascular Intervention}

      \vspace{1.5cm}



      \vspace{4cm} % Adjust space as necessary
      \Large
      \textbf{DEPARTMENT OF}\\
      \vspace{0.1cm} % Adjust line spacing
      \textbf{COMPUTER SCIENCE}

      \vspace{1cm} % Large space as required
      \large
      University of Liverpool\\
      Liverpool L69 3BX

\end{titlepage}
\tableofcontents
\newpage

% Abstract 
\section*{Abstract}
This section should contain a concise summary of the document content.
% Statement of Ethical Compliance
\section*{Statement of Ethical Compliance}
\begin{mdframed}
      \Large % Larger font size for the first two lines
      Data Category: A \\
      Participant Category: 0 \\
      \normalsize % Normal font size for the rest of the text
      I confirm that I have read the ethical guidelines and will follow them during this project. Further details can be found in the relevant sections of this proposal.
\end{mdframed}
\newpage

% Introduction & Background
\section{Introduction \& Background}



\section{Design \& Implementation}

\subsection{Part1: Real-world Model Interaction and Tracking}
% 这一部分集中讨论与现实世界模型的交互与跟踪的设计和实现。

In this part, I used \gls{opencv}\cite{opencv_library} and SciKit-Surgery Augmented Reality\cite{Thompson_SciKit-Surgery_Compact_Libraries_2020} libraries for image
processing and  model tracking. \gls{QT}\cite{QtWebsite} is used to design the graphical interface, and also \gls{vtk}\cite{vtkBook}. OpenCV and SciKit-Surgery libraries help me process images
and track \gls{aruco}\cite{1467495} within video steaming. Qt allows me to create a user interface that ensures  user could more easily change multiple settings, which can enhance the user experience.
VTK and SciKit-Surgery Augmented Reality library are the management of overlay and multi-layer video rendering in my project.
An ArUco Marker Generator is also included, which enables the user to generate different ArUco Markers and save them.
This section will detail the system's design, focusing on System Components and Organization, Data Structures and Algorithms,
User Interface Design, and Design Notation and Diagrams.


\subsubsection{Design}
\begin{enumerate}
      \item \textbf{System Components and Organization}
            \\The project is structured into three primary components, each responsible for distinct functionalities within the system.
            Here’s a detailed breakdown of these components and their organization:
            \begin{enumerate}
                  \item \textbf{Frontend - User Interface}
                        \\The system's user interface is developed using Qt, which is a framework that enables the creation of graphically applications.
                        The main class controlling the UI is \emph{Overlay\_and\_Tracking.py}, which serves as the central hub for user interactions and display functionalities.
                        This class manages the overlay of models on video streaming and provides interactive buttons for users to control various settings, such as model color,
                        video source, models uploading and changing, or adjusting ArUco marker types and sizes, and more.
                  \item \textbf{Backend - Helper Classes}
                        \\The backend is composed of various helper classes, each set to handle specific tasks:
                        \begin{itemize}
                              \item \textit{Image Capture:} The \emph{video\_source.py} class handles the acquisition of video streams from various sources,
                                    including live cameras and recorded media. It is responsible for configuring camera settings, initializing video capture, and video frame cropping to adapted to screen size. This component ensures the reliability and stability of video feed intake.

                              \item \textit{Model Loading:} Managed by \emph{model\_loader.py}, this component is important for the model loading and initializing. It loads ".stl" model files from external files, sets up texture mapping, and prepares the models for real-time overlay. The class also checks for errors in model data to prevent crashes or rendering issues during operation. It also optimizes the structures of model data to enhance rendering efficiency and reduce memory overhead.

                              \item \textit{Model Overlay:} The \emph{overlay\_window.py} is central to integrating 3D models with live or recorded video. With the help of VTK, this module could set up a multi-layered rendering environment where each layer can independently handle elements like video backgrounds, 3D models, or some GUI overlays(In the future, maybe.).

                              \item \textit{Transform Management:} The \emph{transform\_manager.py} class provides a method for managing 4x4 transformation matrices crucial for spatial adjustments of models in 3D space. It stores and retrieves transformations efficiently. And allow it for dynamic modifications of object orientations and positions.

                              \item \textit{ArUco Marker Tracking:} Functionality provided by \emph{arucotracker.py} includes detecting and decoding ArUco markers from the video stream using OpenCV. This module calculates position and orientation of detected marker, and handle the spatial position data for model tracking.
                        \end{itemize}


                  \item \textbf{Additional Component - ArUco Marker Generator}
                        \\An independent component in the system is the ArUco Marker Generator, managed by \emph{Aruco\_Generator.py}.
                        This tool allows users to select and visualize different ArUco markers. Users can also save these markers as separate image files.
            \end{enumerate}
            \paragraph{Organization:\\}
            The system’s architecture is designed to easy maintenance and scalability.
            The modular nature of the helper classes allows for isolated development and testing,
            which enhances the system's robustness and flexibility.
            This organization simplifies development and testing
            and enables the integration of additional functionalities in the future with minimal disruption to the existing system.



      \item \textbf{User Interface Design}
            \begin{enumerate}
                  \item \textbf{Main Menu(Overlay and Tracking)}


                  \item \textbf{ArUco Generator}


                  \item \textbf{Screen Mockups, Sketches, and Screenshots}


            \end{enumerate}


      \item \textbf{Design Notation and Diagrams}
            \begin{enumerate}
                  \item \textbf{Use Case Diagrams}
                        % 描述用例图及其在设计中的应用。

                  \item \textbf{Interaction Diagrams}
                        % 介绍交互图和类图的设计和功能。

                  \item \textbf{Data Flow Diagrams}
                        % 提供系统关键部分的伪代码及数据流图。
            \end{enumerate}
\end{enumerate}

\subsubsection{Implementation}
\begin{enumerate}

      \item \textbf{Backend Implementation}
            % 详细说明系统所使用的数据结构和算法。
            \begin{enumerate}
                  \item \textbf{Pre-processing for video capture and upload}

                  \item \textbf{Feature Data Structures}
                        % 详细说明用于存储从视频中提取的特征(如ArUco标记的位置、角点检测等)的数据结构,可能涉及到空间哈希表或树结构(如KD树、四叉树)。

                  \item \textbf{Image Processing Algorithms in Multi-Layer Video Rendering}
                        % 描述如何存储和处理连续的图像帧,例如使用循环缓冲区或队列来管理实时视频帧。
                        \\ In a video rendering system, video frames need to be dynamically managed to create complex visual effects, such as models and real-time video overlays in this project. This also involved real-time adjustments to the \gls{alpha}\cite{Alpha} and video. A greyscale image with alpha blending of the RGBA stream was used to precisely control transparency and layering effects to enable the superimposition of a layer's frame (e.g. the model) onto the original frame\cite{9979846}. Ensuring overlay accuracy and visual fidelity is crucial for applications such as augmented reality\cite{SETTIMI2022104272}.
                        \\\\
                        In addition, it is necessary to update and align the video frames to the appropriate layers by adjusting the data range according to the size of the incoming video frames. This incorporation of real-time processing improves the continuity of the video image by preventing visual interruptions caused by frame misalignment\cite{Wang}.
                        \\\\
                        I also introduced \gls{aesl} into the multi-layer video rendering system (\emph{Overlay\_and\_Tracking.py}). This enhances image processing, such as when processing video streams involving complex dynamic scenes\cite{7410724}. By dynamically adjusting its smoothing parameter (Alpha), \gls{aes} is able to more accurately adapt to changes in content within a video frame, such as lighting adjustments, scene switching, or object movement, and can reduce visual jitter and blurring due to rapid changes \cite{7410724}.
                        \\\\
                        Compared with the traditional \gls{emal}, the adaptive feature of AES has a greater advantage. \gls{ema}, although fast in processing and low in computational cost, may not be able to adequately adapt its fixed smoothing parameters to real-time changes in the video content in the face of complex scene variations, thus affecting the final image quality\cite{aes,aes2,InvestopediaEMA}.
                        In a multilayer rendering system, combining AES for real-time video transmission and dynamically adjusting the smoothing parameters according to the content differences between the previous and previous frames can maintain the continuity and naturalness of the visual effects, especially when dealing with moving objects and changing backgrounds. In addition, AES's also better handles scenes with large lighting variations, maintaining the balance of colors and shades of light and dark \cite{7298776}.
                        \\\\
                        I have similarly experimented with \gls{cesl}, and while it excels in handling data with clear trends and cyclical variations, its application in video rendering systems may not be as straightforward and effective as AES. Because \gls{ces} is designed to provide a more comprehensive understanding of the multiple influences on the data \cite{ComplexES2018,Complex}, its use in non-predictive applications may lead to overly complex processing and increased computational burden.
                        \\\\
                        Therefore, AES is ultimately used in video rendering systems to respond more directly to real-time changes in video content, reduce visual jitter, and improve the viewing experience.
                        \\\\
                        To summarize, AES can improve image stability and visual quality, as well as enhance the system's responsiveness to environmental changes. By intelligently adjusting processing parameters, AES helps to ensure high efficiency while also adapting to visual jitter or lighting changes that may be encountered with ArUco marker tracking.
                        \paragraph{Code for the AES:}
                        \begin{minted}[frame=single, linenos=true, fontsize=\footnotesize]{python}
class SmoothedTransform:
    def __init__(self, alpha, min_alpha=0.01, max_alpha=0.9):
        """
        Initializes the adaptive smoothing transform class.
        
        Parameters:
        alpha (float): The initial smoothing factor.
        min_alpha (float): The minimum allowable value for alpha 
        to prevent it from becoming too low.
        max_alpha (float): The maximum allowable value for alpha 
        to prevent it from becoming too high.
        """
        self.alpha = alpha
        self.min_alpha = min_alpha
        self.max_alpha = max_alpha
        self.transform = None
      def adjust_alpha(self, new_transform):
            """
            Adjusts the smoothing factor alpha based on the 
            difference between the new transform and the 
            current transform to 
            better adapt to recent data changes.
                                          
            Parameters:
            new_transform (float): The new data point used to 
            update the transform.
            """
            if self.transform is not None:
            # Calculate the absolute difference between the current 
            # and new transforms
            error = abs(new_transform - self.transform)
            # Dynamically adjust alpha based on the error, 
            # inversely scaling it
            self.alpha = max(self.min_alpha, min(self.max_alpha, 
            1 / (1 + error)))
                              
      def update(self, new_transform):
            """
            Updates the current transform with a new data point using
            adaptive exponential smoothing.
                                          
            Parameters:
            new_transform (float): The new data point to incorporate
            into the smoothed data.
                                          
            Returns:
            float: The updated transform value.
            """
            if self.transform is None:
            # If no transform has been set yet, initialize it 
            # with the new transform
            self.transform = new_transform
            else:
            # Adjust alpha based on the new data point
            self.adjust_alpha(new_transform)
            # Apply the adjusted alpha to compute 
            # the new smoothed transform
            self.transform = self.alpha * new_transform + 
            (1 - self.alpha) * self.transform
            return self.transform
                              \end{minted}
                  \item \textbf{Marker Detection and Tracking}
                        % 详述用于检测和追踪ArUco标记的算法,包括图像分割、模式识别和机器学习技术。

                  \item \textbf{Model Positioning and Rendering}
                        % 介绍如何将三维模型精确地放置在虚拟环境中,可能涉及计算几何和物体识别算法。

                  \item \textbf{Optimization Techniques}
                        % 如果涉及到性能瓶颈,可以介绍用于优化上述算法的技术,如多线程处理、GPU加速或使用更高效的算法(如快速傅里叶变换代替传统卷积)。

                  \item \textbf{Model Color change}
                        % 分析关键算法的时间和空间复杂度,讨论在不同操作环境(如不同硬件或软件平台)下的性能表现。

                  \item \textbf{ArUco Generator}

            \end{enumerate}
            \item \textbf{Frontend Implementation}
            % 详细说明系统的用户界面设计和实现。
\end{enumerate}

\subsection{Part2: Endovascular Intervention Simulation}
% 在这里详细介绍关于“Endovascular Intervention Simulation”的设计和实现。

\subsubsection{Design}
% 描述“Endovascular Intervention Simulation”部分的设计细节。

\subsubsection{Implementation}
% 描述“Endovascular Intervention Simulation”部分的具体实现细节。


% Testing & Evaluation
\section{Testing \& Evaluation}

% Project Ethics
\section{Project Ethics}
I have read and abide by the University’s ethical guidelines\cite{UoL_COMP390_2023-24}. The project did not involve direct interaction with human
participants during the design, implementation or evaluation phases. An extensive review of the project scope and methodology
confirmed that no personal data was collected, analyzed or used. In addition, all activities were within the scope of activities
permitted by our ethical guidelines. It was verified with the project supervisor that no customized activities required separate
ethical approval. Therefore, there are no other ethical issues involved in this project.
% Conclusion & Future Work
\section{Conclusion \& Future Work}
\subsection{Conclusion}

\subsection{Future Work}

\section{BCS Criteria \& Self-Reflection}
This section will be used to state that my project met the six outcomes expected by the Chartered Institute of Information Technology\cite{BCS2020}.
I will focus on illustrating an ability to self-manage a significant piece of work and the critical self-evaluation component.
\subsection{An Ability to Apply Practical and Analytical Skills}
The project has demonstrated the practical and analytical skills I have learnt during my time at university. Throughout the degree I have gained a deeper understanding of programming languages such as Python, C\# , Java and C\+\+ and have gradually begun to experiment with them. The theoretical and practical foundations of these languages have been key in enabling me to achieve the complex functionality required for development and realisation projects. For example, in the Part1: Real-world Model Interaction and Tracking section of my project, which was written entirely in Python, there was a high level of theoretical and practical demand for the Python language. In my Part2: Endovascular Intervention Simulation, I needed to acquire and apply knowledge such as the application of Unity and the development and application of the C\# language that I had learnt in my degree programme. These technical skills were acquired and refined through a careful learning process and were directly applied to the project, which dealt with the development of a real model interaction and tracking system and the development of a Unity-based Endovascular Intervention Simulation.
\\\\
In developing Part1: Real-world Model Interaction and Tracking and Part2: Endovascular Intervention Simulation, I have also made extensive use of the Artificial Intelligence, Game Development and Computer Vision knowledge that I have learnt on my degree course.
\\\\
For Part1: Real-world Model Interaction and Tracking, I utilised the techniques learnt in the Computer Vision course to process the images and used the OpenCV package usage learnt in the course to implement the tracking of the ArUco markers. By utilising the image processing and tracking capabilities of OpenCV, accurate model interaction in complex environments is carried out in practice.
\\\\
In Part2: Endovascular Intervention Simulation, I used my knowledge of game development to develop a Unity project, using \gls{blender} to modify and optimise the model, and applying Unity techniques to ensure that Rope interacts with the blood vessels.
\\\\
This project, dedicated to the development of a realistic simulation used for endovascular interventions in a virtual reality environment, emphasised my ability to integrate practical skills and theoretical insights, demonstrating a deep understanding of the technical and theoretical aspects I have learnt during the course.
\\\\
Overall, this project clearly demonstrated my ability to apply the analytical and practical skills acquired during my degree programme. It also demonstrated my understanding and use of complex programming techniques and frameworks, as well as my ability to use multidisciplinary knowledge to cross-cut problem solving. Through this project, I have accomplished the ability to translate my learning into practical applications in the real world.



\subsection{Innovation and/or Creativity}
There are some innovations in the field of medical simulation technology in my project, especially Part2: Endovascular Intervention Simulation, The aim of Part2 is to create one of the few open source endovascular intervention simulation projects in the field to make training tools more accessible to the medical community. The project combines traditional surgical simulation with augmented reality/virtual reality technology to make surgical simulation procedures visual and easy to practice. Compared to traditional simulations that are limited to 2D screen displays, this approach uses virtual reality to present surgical simulations in 3D space, which not only enhances the realism of the simulation and interactions, but also allows users to experience and understand the steps involved in the surgery more clearly by allowing them to interact with the simulated environment in a more intuitive and natural way.
\\\\
The project uses an open source framework (\gls{mrtk}) and the integration of Augmented Reality/Virtual Reality (AR/VR) technology to provide a practical and innovative application for educational tools in the medical field.


\subsection{Synthesis of Information, Ideas, and Practices}
This project integrates development tools and theoretical principles from different fields to design the open source Endovascular Intervention Simulation to provide a convenient tool for medical surgery simulation or surgical training.
\\\\
In the first part of the project, Real-world Model Interaction and Tracking, open source development tools such as OpenCV, VTK and SciKit-Surgery were used. The use of these powerful tools allows me to design user-friendly graphical interfaces or to enhance the model rendering capabilities and real-world model tracking capabilities of my application. OpenCV provides a rich set of image processing tools to help me perform complex image processing and tracking, while VTK and SciKit-Surgery provide powerful tools for medical impacts, such as multilayer image rendering and overlays. This can help me combine tracking capabilities with augmented reality to create a more interactive virtual reality system for users. This section combines knowledge, tools and ideas from various fields to create a high quality solution.
\\\\
Part 2: Endovascular Intervention Simulation Developing an application on \gls{holo} using \gls{mrtk} translates the theoretical knowledge I have learnt in my school course such as C\# and Unity development into a practical solution. The development utilised model modification and knowledge related to Unity development, C\# development, etc. to transform Endovascular Intervention Simulation, which is traditionally limited to a flat display, into a virtual reality simulation with immersive, interactive features. This part of the development demonstrates how AR application development techniques and Unity development can be combined to provide a virtual reality surgical simulation with multiple functionalities.
\\\\
Both parts of the project exemplify how technical and theoretical knowledge from the fields of computer vision, artificial intelligence and software development can be applied to create effective and innovative medical training tools.
\subsection{Meeting a Real Need in a Wider Context}
Both parts of the project, Part1: Real-world Model Interaction and Tracking and Part2: Endovascular Intervention Simulation, address some of the broader needs in the medical field.
\\\\
For Part1: Real-world Model Interaction and Trac, current market systems usually lack user-friendly graphical interfaces, and features such as model colour, selection of different ArUco markers, and resizing are lacking or incomplete, which can cause some degree of difficulty for users. For Part1 the project adds a graphical interface and provides a variety of modifiable parameters to optimise these shortcomings, making the software less difficult to use and better adapted to the needs of a wide range of scenarios.
\\\\
Part2 considers the lack of open source endovascular intervention simulation in the market and the fact that most existing simulation tools are limited to 2D planar presentation and cannot meet the complex 3D visual and operational needs. The aim is to develop an open source platform that supports immersive 3D simulation, AR/VR and other functions. The system can support 3D simulation in AR/VR (deployed in Microsoft \gls{holo}), and by lowering the barrier to use through more intuitive and simple controls, it can be used in the future to allow healthcare professionals or non-professionals alike to experience or learn surgical skills. This simulation tool can not only be used for professional training, but also meets the need for telemedicine services that can provide remote diagnosis and treatment in the future.
\\\\
Overall, it is planned that these two components will be combined in future work, which can meet the needs for simulation of surgical training simulation for simplicity, remote operation, and 3D highly experiential simulation. In the future it may be possible to expand into more areas to meet a wider range of needs, such as providing an immersive experience of Endovascular Intervention Simulation for lay people.

\subsection{An Ability to Self-Manage a Significant Piece of Work}
In my project I demonstrated the ability to self-manage a significant piece of work, but it was partially flawed. The project consisted of two widely differing parts, which added to the difficulty of managing the project as a whole, and in order to keep both of the major parts of the project accurately planned and executed, I used a variety of tools and methods to ensure that the project ran smoothly.
\\\\
Firstly, I used a number of time management tools to map out the timeline of the project, including key milestones, time required, and deadlines for each progress module. I created some Gantt charts and schedules to help me monitor the progress of the project and try my best to make sure that the tasks in each phase are completed on time. In this way I could get a clear picture of the overall progress of the project and adjust the plan as much as possible in time to cope with possible delays. However, as this was my second time working on a larger volume project (the last time was COMP208 Group Project), I was not able to be very perfect in creating the schedule and Gantt chart, and some mistakes were made.
\\\\
Secondly, in terms of project management, I used \gls{git} to maintain version control of the project and writing between team members. I used GitHub to effectively track code updates and backups, as well as to enable team members to view the latest progress of the project in real time and provide feedback. GitHub's version control and backup features have many times saved errors caused by mistakes, effectively avoiding many accidents. Using this open source platform has helped me to manage a major task and increase the efficiency of multi-person collaboration.
\\\\
In addition, in order to control the development progress and quality, I also hold weekly progress meetings with my team members and supervisor to report the progress of this week's work and discuss and plan the next work. In these meetings, I can get sufficient feedback to help me modify and optimise my previous work, and make reasonable planning and arrangement for the next work. This regular reporting and discussion has ensured that the project has developed according to the set objectives.
\\\\
I also focus on stage-by-stage problem analysis and risk management during project implementation. Whenever the project progresses to a certain stage, I will review the previous work, check and improve any possible problems in the completed work, and make sure that the previous work is accurate before proceeding to the next stage1. This is a good way to ensure that I make fewer mistakes when managing a large volume of work.
\\\\
Whilst I have adopted a variety of methods during the project management process to ensure that the project is executed efficiently and to a high quality as planned, I still have some shortcomings in my ability to self-manage a significant piece of work. For example, although I produced a Gantt chart and schedule to monitor the progress of the project, at the beginning of the project I did not properly consider the time required for some parts and the difficulties I may have encountered, for example, I encountered great difficulties in carrying out the initial design and import of the Rope in Part2: Endovascular Intervention Simulation, etc., and the rate of progress was not as fast as expected. This resulted in the project progressing at a much slower pace than expected and led to the project being put on hold for some time. In addition, although we had weekly meetings, some of them were of minimal effect. These meetings usually took place when the project was experiencing some major difficulties, and in these meetings the solutions to the problems and the planning for the next phase of the project progress were not discussed very effectively.
\\\\
In addition, I had problems with teamwork when using GitHub for project management. Poor documentation, different operating systems used by each member, conflicting versions of various software packages, and GitHub's file size limitations for uploading files caused many difficulties for team members when sharing through GitHub.
\\\\
Overall, this project demonstrated my ability to self-manage a significant piece of work, but it also demonstrated my shortcomings in some of these areas. This project gave me a great opportunity to optimise my ability to manage projects, such as time planning skills and communication with team members, as well as making me realise what I need to learn and improve in project management.
\subsection{Critical Self-Evaluation of the Process}
In my projects, Part1: Real-world Model Interaction and Tracking and Part2: Endovascular Intervention Simulation, although both have been accomplished, there have been many challenges and difficulties in the development process, and so far there have been some shortcomings. Through in-depth critical self-evaluation, I was able to comprehensively analyse the successes and shortcomings of the project, as well as gain experience and lessons learned.
\\\\
First of all, the functionality of my project Part1 is relatively complete, which can effectively implement real-time model overlay and ArUco marker tracking, and complete the basic function of my plan. At the same time, I also added extra features such as ArUco icon parameter tuning and ArUco icon generator. However, there are some technical limitations in this part, mainly in platform compatibility. Currently, the system only runs on the Linux platform and has not yet implemented support for other operating systems such as Windows or macOS, nor has it been able to complete deployment on VR/AR devices such as HoloLens. This limitation may have impacted the project's widespread adoption. In this regard, I believe that my Part1 met the requirements of my plan, but still needs to be improved and extended in terms of compatibility.
\\\\
For Part2, the development process encountered significant technical challenges, especially during the stages of designing the Rope and developing the method of interaction between the Rope and the vessel wall. These technical issues led to delays in the development progress and the final product implemented only the most basic functionality and did not achieve the level of Rope-vessel wall interaction required at the beginning of the design. This difficulty stemmed from my lack of skill in using development tools such as Unity and Blender, and my underestimation of the complexity of the interaction logic of physical simulation and model interaction. Nonetheless, the development process has greatly strengthened my technical skills in 3D modelling, Unity development and physics simulation simulation.
\\\\
By critically reflecting on and analysing these issues, I realised that I should plan better in the upfront technical assessment and time management phases when undertaking future project management. This includes analysing in detail the technical difficulty and time required for each development phase during the project planning stage, as well as being prepared in advance to deal with unforeseen circumstances as they occur. In addition, it is also important to communicate more with team members and supervisors at the technical level during the project to accelerate the speed of breaking through the development challenges.
\\\\
Overall, the development process of this project was full of difficulties and challenges, but it also strengthened my technical level, project management skills and the ability to solve unknown problems. Through this project, I was able to improve my professional skills in a variety of different technical areas, as well as develop my ability to effectively self-manage and work in a team on projects with complex environments. Through this critical self-assessment and reflection, I was able to better identify and improve on my shortcomings in my work and prepare myself for the greater challenges I may face in the future.

% References (The bibliography will be printed here)
\printbibliography
\printglossaries

\end{document}
